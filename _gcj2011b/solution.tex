\documentclass[UTF8]{ctexart}
	\title{Rains Over Atlantis题解}
	\author{唐适之}
	\date{}
	\newcommand{\myparagraph}[1]{\paragraph{#1}\mbox{}\\}
	\usepackage[top=1in, bottom=1in, left=1.25in, right=1.25in]{geometry}
	\usepackage{enumitem}

\begin{document}
	
	\maketitle
	
	\myparagraph{算法1}
		
		最简单的方法是模拟,每天先算出每个格子的水平面:从低到高枚举水平面,再FloodFill整个地图看有哪些区域的水不能流出去;然后扫描每个格子求出其被侵蚀的高度,建出下一天的地图,重复执行。最多只有maxHeight天。时间复杂度$O(maxHeight \cdot (HW)^2)$,空间复杂度$O(HW)$,期望得分20分。
		
	\myparagraph{算法2}
	
		依然是模拟,算水平面可以用Dijkstra算法,即从海面出发到某点路径最高点的最小值即该点的水平面。不建议使用SPFA,因为它在网格图上表现不好。时间复杂度$O(maxHeight \cdot HW \cdot \log_2(HW))$,空间复杂度$O(HW)$,期望得分40分。
	
	\myparagraph{算法3}
	
		海拔高度很大而M很小时直接模拟就不能解决问题了,所以我们需要同时计算多天的侵蚀。当下列条件同时满足,就可以同时计算多天:
		
		\begin{itemize}[leftmargin=15mm]
			\item 所有未被淹没的格子都以最大速度被侵蚀。
			\item 所有的水面都在以最大速度下降。
			\item 未出现被淹没的格子露出水面的过程。
		\end{itemize}
		
		像下面这样同时计算多天:
		
		\begin{enumerate}[leftmargin=15mm]
			\item 计算每个格子的水平面。
			\item 判断每个未被淹没的格子是否以M的速度被侵蚀,如果不是就只计算一天。\textbf{注意我们可以让格子的海拔降到负数},即假定地图外的海拔是无穷小,这样不会改变结果,并会大大简化问题。
			\item 如果每个未被淹没的格子都是以M的速度被侵蚀,那么每个“湖”的水平面也会以M的速度下降。因为一个“湖”是由至少一个在湖边上未被淹没的格子决定高度的,而这个格子的高度正以M的速度下降。所以某些被淹没的格子露出水面之前的侵蚀都可以一起计算。
			\item 由此我们找出被淹没格子水深的最小值,就可以算出有多少天可以一起计算。
			\item 如果没有任何一个格子在水下,就还剩$\lceil \mbox{最高海拔}/M \rceil$天就侵蚀完了。
		\end{enumerate}
		
		注意每次同时计算后都会有一个被淹没的格子露出水面,而显然露出水面的格子不会再被淹没,所以最多执行$HW$次同时计算。(事实上这个数字会更小,因为地图边缘的格子不会被淹没)
		
		现在我们就要算出最多要经过多少次普通的一天一天地计算后,才能进行同时计算。定义一个辅助的图来进行说明:每个未被淹没的格子对应图上的一个节点,每个“湖”的所有格子对应图上的一个高度为“湖”的水平面的节点。节点S和节点T间有连边当且仅当S对应的任意一个格子与T对应的任何一个格子有公共边。称一个节点的父节点为与其相邻的高度最低的节点。这样我们就有了一棵根在外海中的树。我们定义一条向海拔高的方向走的的树上路径为确定路径,当且仅当这条路径上每条边的高度差都为M。我们能证明每天过后都能发生如下的事件之一:
		
		\begin{itemize}[leftmargin=15mm]
			\item 有被淹没的区域露出水面了(这样执行同时计算的次数就会减少)
			\item 确定路径上节点增多了
			\item 所有的节点都在确定路径上了
		\end{itemize}
		
		原因如下:考虑任意一条确定路径。假设节点的分布维持不变(即陆地还是陆地,湖还是湖),这条路径在一天后还是确定路径,因为其上的每个节点都在以M的速度降低。考虑任意一个不在确定路径上但它父节点在确定路径上的节点(注意前面为了方便把海平面定义为无穷低,海平面在确定路径上,如果不存在这样的节点,那么所有节点都在确定路径上了),一天后此节点的高度降到了它父节点的高度,而它父节点的高度下降了M,所以现在它和它父节点的高差也变成了M,它也加入了确定路径。
		
		如果所有的节点都在确定路径上,我们就能进行同时计算了。每次同时计算前最多有$HW$次路径上节点的增多,这样每一个被淹没的格子露出水面前最多有$HW$次普通计算和一次同时计算,总计$(HW)^2$次普通计算和$HW$次同时计算,总时间复杂度为$O((HW)^3 \log_2(HW))$,空间复杂度$O(HW)$,期望得分100分。
		
		\emph{
			注:可以像下图这样,在M=2时,通过一条曲折、不自交、长度为$O(WH)$的路径达到算法的时间上界:
		}
		
		\begin{center}\fbox{\shortstack{
X X X X X X X \\
P P P P P P X \\
X X X X X P X \\
X P P P P P X \\
X P X X X X X \\
X P P P P P X \\
X X X X X X X
}}\end{center}
		
		\emph{
			其中X表示非常大的数,P表示这条路径。这样在整个算法的过程中,上面提到的树一直仅是这条路径。
		}
		
		\emph{
			这条路径取如下海拔:A, A-2L-1, A+1, A-4L-1, A+1, A-6L-1, A+1, A-8L-1, 等,其中L是这条路径的长度。不难发现,确定路径要一个湖一个湖地向下形成,而每次同时计算后湖的数量只会减少一个,确定路径却要重新形成。
		}
		
\end{document}