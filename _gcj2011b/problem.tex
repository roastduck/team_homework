\documentclass[UTF8]{ctexart}
	\title{Rains Over Atlantis}
	\author{唐适之}
	\date{时间限制2s,空间限制256M}
	\newcommand{\myparagraph}[1]{\paragraph{#1}\mbox{}\\}
	\usepackage[top=1in, bottom=1in, left=1.25in, right=1.25in]{geometry}

\begin{document}
	
	\maketitle
	
	\myparagraph{问题描述}
		
		大雨无情地侵蚀着亚特兰蒂斯的土地,并终将毁灭它。为了组织人民撤离,你想知道亚特兰蒂斯多快就会毁灭殆尽。
		
		你有一幅亚特兰蒂斯的地图。地图是一个矩形的网格,每个格子里有一个整数表示该位置海拔为多少米。网格外面是大海,海拔为零。所有为海拔零的格子都是水域,所有海拔大于零的格子都是陆地,没有海拔小于零的区域。
		
		如果两个格子有公共边,水就能从高的格子流向低的格子。如果两个有公共边的格子一样高,水能向任意一边流。
		
		因为雨非常大,所以如果一个格子里的水流不走,就会积在那里,直到水平面足够高使其得以流走。地图外面的大海可以接收任意多的水。举个例子,比如下面这幅地图:
		
		\begin{center}\fbox{\shortstack{5 9 9 9 9 9 \\ 0 8 9 0 2 5 \\ 3 9 9 9 9 9}}\end{center}
		
		低洼地区会积水。我们把积水地区的海拔加上水深称为\emph{水平面},上面地图的水平面将会是:
		
		\begin{center}\fbox{\shortstack{5 9 9 9 9 9 \\ 0 8 9 \textbf{5 5} 5 \\ 3 9 9 9 9 9}}\end{center}
		
		注意中间地域的0,尽管它是水域,因为它不与外界相连,所以它也会积水。边界上的0与外界相连,所以来自8的水可以经它流走。
		
		水流动的方向决定于水平面的高低。如果与一个区域相邻(有公共边)有若干个区域的水平面都比它低,那么它的水就会流向其中最低的一个。如果最低的也有多个,流向哪里没有所谓,这在下文会提到。
		
		现在流水侵蚀开始了。每天,一个格子被侵蚀掉多少——海拔降低多少——决定于水怎么流过它。如果水从S流到与S相邻的T,那么S的海拔将降低min(S的水平面-T的水平面,M)。所有的侵蚀都在同一时间——一天结束的时候——发生。例如,M=5时,上面的地图描绘的土地就会被侵蚀成下面这个样子:
		
		\begin{center}\fbox{\shortstack{0 4 4 4 4 4 \\ 0 3 5 0 2 0 \\ 0 4 4 4 4 4}}\end{center}
		
		一天的侵蚀过后,多余的水会流走:当一片区域的水平面高于与它相邻的区域的水平面时,水就会从高处流到低处,直到两个区域水面相平。水依然回像第一天那样积累。第二天,水平面变成:
		
		\begin{center}\fbox{\shortstack{0 4 4 4 4 4 \\ 0 3 5 \textbf{2} 2 0 \\ 0 4 4 4 4 4}}\end{center}
		
		又过了一天的侵蚀,地图又变成下面这样:
		
		\begin{center}\fbox{\shortstack{0 0 0 0 0 0 \\ 0 0 2 0 0 0 \\ 0 0 0 0 0 0}}\end{center}
		
		……这时亚特兰蒂斯的居民就要紧急疏散了。你的任务是计算要过多少天亚特兰蒂斯的海拔高度会全变成0。
		
	\myparagraph{输入格式}
		
		第一行读入一个整数T,表示接下来有T组数据。每组数据由包含三个整数H、W、M的一行开始,这三个整数分别表示地图的长、宽和一天最大的侵蚀高度。接下来H行每行W个整数,其中的第i行第j个数表示地图中(i,j)格子的高度。
		
	\myparagraph{输出格式}
	
		对于每组测试数据,输出一行“Case \#x: y”(不包含引号),其中x表示测试数据的编号,y表示多少天就能侵蚀完整个亚特兰蒂斯。
		
	\myparagraph{样例输入}
	
		\begin{verbatim}
			2
			3 6 5
			5 9 9 9 9 9
			0 8 9 0 2 5
			3 9 9 9 9 9
			3 6 3
			3 8 10 11 10 8
			7 5 2 12 8 8
			6 9 11 9 8 4
		\end{verbatim}
		
	\myparagraph{样例输出}

		\begin{verbatim}
			Case #1: 3
			Case #2: 5
		\end{verbatim}
	
	\myparagraph{数据规模和约定}
	
		共20个测试点,每个点5分。
		
		测试点1$\scriptsize{\sim}$4:1≤T≤10, 1≤H,W≤10, 1≤M≤100, 0≤所有海拔≤100。
		
		测试点5$\scriptsize{\sim}$8:1≤T≤50, 1≤H,W≤20, 1≤M≤100, 0≤所有海拔≤100。
		
		测试点9$\scriptsize{\sim}$20:1≤T≤10, 1≤H,W≤20, 1≤M≤$10^{15}$, 0≤所有海拔≤$10^{15}$。
\end{document}